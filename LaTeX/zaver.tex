\chapter{Závěr}
    Cílem této práce bylo navrhnout, zkonstruovat a naprogramovat meteostanici, která měří základní meteorologické veličiny. Meteostanice měří vnitřní a~venkovní teplotu, vnitřní a venkovní vlhkost, koncentraci $CO_2$, atmosférický tlak, rychlost větru a stav akumulátoru. Naměřená data zpracovává, zapisuje na paměťovou kartu a zobrazuje na displeji i webserveru. Splňuje tedy zadání. S~výsledkem jsem spokojený, avšak je tu dost věcí, které bych chtěl do budoucna přidat či vylepšit. Po tomto školním roce plánuji mít meteostanici doma na zahradě a dále ji vyvíjet po hardwarové i softwarové stránce. Do budoucna bych chtěl přidat například část na měření úhrnu srážek. Tento senzor funguje na principu kolébky, do které stéká voda z~nálevky. Když se půlka kolébky naplní, překlopí se a plní se druhá půlka. Při každém překlopení by se magnet na kolébce přiblížil ke stejnému senzoru jako v~anemometru. Podle počtu pulzů, času, prostoru nálevky a objemu kolébky by bylo možné vypočítat aktuální úhrn srážek za nějaký čas. Dále bych rád přidal měření směru větru a naprogramoval LUT (Look Up Table) pro zpřesnění měřených hodnot ADC. Práce na tomto projektu mi zcela jistě přinesla mnoho zkušeností, ať už v elektronice, programování, nebo 3D návrhu. Zlepšil jsem se v~modelování i~v~samotném návrhu a konstruování. I když to možná nevypadá, vývojem této meteostanice jsem strávil neskutečné množství času. A tak doufám a věřím, že jsem při tom zlepšil své schopnosti a naučil se něco nového a užitečného.