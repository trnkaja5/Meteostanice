\chapter{Vyhodnocení} \label{Vyhodnoceni}
    Obě jednotky meteostanice jsem testoval v průběhu výroby. vnitřní i venkovní jednotku jsem několikrát nechal společně běžet celý den, někdy i přes víkend. V průběhu jsem kontroloval zápis na paměťovou kartu a vylepšoval vzhled a vlastnosti webserveru. Kalibraci anemometru jsem prováděl pomocí vypůjčeného anemometru, který byl součástí sady Voltcraft UM5/1 100. Měření jsem provedl vícekrát a pokaždé s jinou rychlostí větru. Při každém měření vyšla odchylka trochu jinak. Přibližně se z toho ale dá určit konstanta 2,7 pro korekci rychlosti větru. Během měření se tedy naměřená rychlost otáčení vrtulky vynásobí touto konstantou a vyjde přibližná rychlost větru.

    Venkovní i vnitřní jednotka fungují podle zadání, dosah rádiové komunikace je dostatečný. Akumulátor se při chodu stíhá dobíjet solárním panelem. Vnitřní jednotka zpracovává a zobrazuje data. Anemometr se točí nad má očekávání, ložisko má velmi nízké tření. S radiačním krytem jsem také spokojený, myslím, že plní svůj účel. I přes všechny tyto úspěchy je tu však samozřejmě několik věcí, které by se daly vylepšit nebo přidat. Z~těch hlavních  mi na meteostanici chybí měření úhrnu srážek a měření směru větru. Dále by se také dalo vylepšit zpracování i zobrazení naměřených dat (například nějaké grafy). 

    Do obou jednotek jsem chtěl přidat možnost připojit NTC teplotní čidlo, kdyby bylo potřeba měřit na dalším místě. Hlavně u vnitřní jednotky by to bylo užitečné. Narazil jsem ale na problém s AD převodníkem. U obou mikrokontrolerů ESP32 je velmi nepřesný. Šum pomohl vyřešit kondenzátor a průměrování několika naměřených hodnot. ADC ale není po celém rozsahu dokonale lineární. Při měření napětí akumulátoru to není takový problém. Akumulátor je připojený přes odporový dělič 1:2 a napětí na GPIO pinu tedy nabývá hodnot od 1,85 do 2,1 V, což je ještě v prostředním pásmu, kde je charakteristika ADC lineární. I tak je potřeba k hodnotě přičíst konstantu, protože je zde mírný offset. při měření odporu NTC čidla by se napětí pohybovalo ve větším rozsahu a jednoduchá konstanta pro korekci už by nestačila. Řešením je vytvořit takzvanou Look Up Table, která by každou hodnotu z ADC ($2^{12}$) přiřadila ke správné hodnotě napětí.