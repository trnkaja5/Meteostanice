\cleardoublepage
\chapter*{Úvod}

    V tomto projektu se budu zabývat návrhem a výrobou meteostanice, která bude měřit vnitřní a venkovní teplotu a vlhkost, koncentraci $CO_2$, dále atmosférický tlak a rychlost větru. Naměřená data zobrazí na displeji a bude je pravidelně ukládat na paměťovou kartu. Na mikrokontroleru vnitřní jednotky také poběží webserver, kde se zobrazí naměřená data a stav akumulátoru ve venkovní jednotce.
    
    Výstupem práce bude funkční meteostanice připravená pro každodenní dlouhodobé použití. Vybral jsem si vytvoření tohoto projektu kvůli zájmu o~monitorování počasí a jeho průběhu během dne, týdne, měsíce či roku. Věřím, že bude přínosný nejen pro mě.
    
    Tento projekt je aktuální díky své schopnosti poskytovat důležitá data o počasí. V současné době, kdy jsme svědky častých extrémních změn v klimatu, je můj výrobek užitečný v každodenním životě. Bez znalosti aktuálního počasí přímo u Vás doma není možné efektivně, kvalitně a pohodlně využívat svůj čas. Navíc pomůže lépe porozumět aktuálním klimatickým podmínkám a dlouhodobým trendům.
    
    V kapitole~\ref{ElCast} o elektronické části se budu věnovat návrhu zapojení senzorů a prvků v obou jednotkách, návrhu plošného spoje a jeho výrobě. V kapitole~\ref{KonstCast}, věnující se konstrukční části, se zaměřím na modelování mechanických částí projektu a jejich výrobu. Kapitola~\ref{PrgCast} se věnuje programování. Zde bude nejdůležitější zprovoznit všechny senzory, zajistit komunikaci mezi venkovní a vnitřní jednotkou a zpřehlednit data uživateli. Nemalá část bude patřit také zobrazování dat na displeji a webserveru a jejich export na paměťovou kartu.