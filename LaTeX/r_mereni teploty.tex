\chapter{Měření teploty}

    Měření teploty vzduchu je pro meteorologii a předpověď počasí velice důležité. Teplota vzduchu se měří pomocí teploměru. Teploměry se dělí na skupiny: dilatační teploměry, elektrické teploměry, speciální teploměry a bezdotykové teploměry. Skupina dilatačních teploměrů se dále dělí dle typu na: plynové, tenzní, kapalinové, kovové. Z nichž nejznámější jsou kapalinové, které měří objem kapaliny, jenž je závislý na teplotě. Sem patří například lihový, nebo pro profesionální měření častěji používaný rtuťový teploměr. Kapalinové teploměry jsou spolehlivé, přesné a levné, avšak jejich nevýhodou je křehkost a obtížnost dálkového přenosu údaje~\cite{Mereniteploty}. Další skupinou jsou elektrické teploměry, které se dále dělí na termoelektrické, odporové kovové, odporové polovodičové, diodové. Výstupem elektrických teploměrů je analogová hodnota v podobě elektrického napětí, které je na teplotě závislé podle určitého vztahu. Pro zpracování údaje digitálním zařízením se používají A/D převodníky.
    
    \section{Termoelektrické}
        Termoelektrické teploměry měří teplotu pomocí termoelektrických článků, které jsou vyrobeny z vodičů ze dvou různých kovů, které jsou vodivě spojeny. Při rozdílu teplot obou materiálů mezi nimi vzniká termoelektrické napětí. Zjednodušeně lze závislost termoelektrického napětí na teplotě vyjádřit lineárním vztahem. Jednotlivé termočlánky se označují velkými písmeny. Nejpoužívanější jsou články s označením T, J, X, S~\cite{Mereniteploty}.
        
    \section{Odporové}
        Odporové teploměry se dělí na kovové a polovodičové. Kovové teploměry se vyrábí výhradně z čistých kovů. Vhodné materiály jsou například platina, nikl a měď, z nichž nejpoužívanější je díky jednodušší výrobě a její fyzické i chemické stálosti platina. Na materiálu také záleží kvůli teplotnímu součiniteli. Ten by ideálně měl být stálý a co největší. Platina je výhodná také díky jejímu z ostatních materiálů nejvyššímu měřícímu rozsahu (-200 až 850 °C)~\cite{Mereniteploty}.
        
        Druhou skupinou odporových teploměrů jsou polovodičové, které se dále dělí na NTC-termistory, PTC-termistory a monokrystalické senzory teploty. Asi nejznámějším typem jsou NTC-termistory. Vyrábí se spékáním oxidů a důležité je, že závislost odporu na teplotě není lineární. Odpor s rostoucí teplotou klesá podle určitého vztahu~\cite{Mereniteploty}. Pro určení teploty podle elektrického odporu NTC-termistoru je potřeba znát β koeficient a daný odpor termistoru při teplotě 25 °C. β koeficient lze vypočítat, pokud známe odpor termistoru alespoň při dvou různých teplotách.
